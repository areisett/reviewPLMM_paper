\documentclass{article}
\usepackage[left=0.5in,right=0.5in,bottom=0.75in,top=0.75in]{geometry}
\usepackage{amsmath}
\usepackage{amssymb}
\usepackage{natbib}
\usepackage{color}
\providecommand{\note}[1]{\textcolor{red}{#1}}

\begin{document}

\subsubsection*{Response to comments by Reviewer \#1}

Thank you very much for your careful reading of the paper and your insightful comments. You raise a number of good points and we have modified the manuscript considerably based upon your feedback.  In our opinion, these revisions have very much improved upon the original submission.

\begin{enumerate}

\item \emph{This is a nicely conducted review/educational article on an important topic. My main ``complaint'' is that it is not ``sufficiently friendly''.  The article directly jumps into a statistical question (and solutions). There is insufficient discussion/description on the application scenarios. There can/should be clearer descriptions on when the reviewed approaches can/cannot be applied.}

  You raise a good point.  We have tried to improve upon this aspect of the paper by rewriting the introduction with the application, rather than the statistical methodology, in mind.  In addition, we added a new section (more details below) illustrating the method on a case study involving a GWAS with a mixed population (white/black/hispanic).
  
\item \emph{In the discussion of methods, there are similar concerns. For example, for some or most statisticians, Lasso is not a new concept. But I can bet that not all practitioners know what Lasso does. I acknowledge that this article cannot and should not review Lasso and other penalizations. But still, it is worthwhile briefly discussing what the key methodological components can do and providing references.}

  This is an excellent point.  We have now added a new section, 3.1, giving a brief overview of the lasso, as well as references to learn more about it.

\item \emph{Simulation is conducted and is informative. However, for practitioners, it is of interest to see what kind of practical impact the approaches may have. Data examples are strongly recommended.}

  We have added a new section (Section 5) that contains a ``case study'' GWAS involving real data.  For a variety of reasons, we chose to use a synthetic phenotype, but all of the genetic data is real.  This allows us to inspect the accuracy of these methods, as the true causal SNPs are known, and did indeed reveal some additional insights that the smaller-scale simulations did not.

\item \emph{Related to the above points: how can someone not savvy in statistics/penalization etc. use these approaches? Any simple software? I can see some scattered information for example in Discussion. One possibility is to have a separate subsection or table for software. Application examples too.}

  We have added a new section (Section 3.8) that specifically addresses the availability of software implementing these methods.  As far as whether the software is ``simple'', we tried to make our R package, \texttt{penalizedLMM}, as user-friendly as possible, with convenient functions for importing data from PLINK files and preprocessing (e.g., carrying out genomic imputation prior to fitting).  We hope that this makes these models relatively easy to fit and use for practitioners.

\item \emph{I understand that the authors’ focus is on linear model. But, many if not most genetic studies do not have a continuous outcome and linear model. Can the models/approaches be extended to other outcomes/models with reasonable effort? Either way, it is worth knowing.}

  This is an excellent point.  We have expanded our discussion of this issue in section 2.3, which now contains the following paragraph:

  \begin{quote}
    aaa
  \end{quote}

\item \emph{This may seem contradicting the above points: for a few statisticians, it may be of interest to know more about the statistical properties, especially under high-dimensional settings. Brief discussions (maybe a paragraph? Or a subsection?) on the statistical properties may be informative.}

  It's possible that we don't entirely understand what you mean here, as this was the point of all the simulation studies: to better understand the statistical properties of these methods.  Perhaps you mean the theoretical, mathematical properties?  If so, unfortunately, not much is known.  We have expanded Section 3.7.4, which discusses tuning parameter selection (this was suggested by another reviewer).  In this section, we do discuss some of the work that has been done, including work on inference and estimation of false discovery rates, and provide references.  We also ``note that our theoretical understanding remains relatively poor and there is no consensus on
the optimal way to select $\lambda$ in this case (Wauthier et al., 2013).''  We hope that these modifications help to address this issue.

\end{enumerate}

\newpage

\subsubsection*{Response to comments by Reviewer \#2}

Thank you very much for your detailed reading of the paper and your insightful comments. You raise a number of excellent points and we have modified the manuscript in several places based upon your feedback.  In our opinion, these revisions have nicely improved upon the original submission.

\begin{enumerate}

\item \emph{This is a very insightful paper as to the performance of lasso LMM methods in GWAS analyses and provides nice intuition using theoretical derivations on the interplay between population structure and environmental heterogeneity. The paper could highly benefit from using more realistic simulation scenarios reflecting standard GWAS (increase the number of SNPs in model which is more in the 100,000s in current GWASes) \ldots}

  This is an excellent point, and shares some concerns with a point made by reviewer \#1, who wished to see a real data example.  We attempted to address both of these concerns with a ``case study'' (Section 5), which uses genotype data from a real GWAS (with 526,266 SNPs) but with a synthetic phenotype, which allows us to assess the accuracy of the various methods in terms of identifying the causal SNPs.  Indeed, this new setting did reveal some additional insights that the smaller-scale simulations did not.

  \item \emph{\ldots as well as provide some assessment of how robust the results would be to different lasso penalty parameters (as in practice we would not have a priori an optimal value).}

  Another excellent point.  We have expanded Section 3.7.4 to include a much more thorough discussion of the problem of tuning parameter selection.  In addition, all of the results in Section 5 (the new GWAS case study) are presented over the entire path, so that readers can see how the statistical properties of the estimates change as $\lambda$ changes.

\item \emph{In the captions of Figs. 1 and Table 1, it is not clear how many replicates are used to estimate these quantities (e.g. all p=1000 SNPs used in simulations or only the 50 SNPs selected by lasso?).}

  Thank you for pointing this out.  The number of replications is now explicitly stated in all of the figure and table legends.  Also, we have added this sentence at the beginning of Section 4.1, after defining MSE: ``In particular, note that this calculation includes all $p$ coefficient values, regardless of whether or not they were selected by the model.''

\item \emph{For Figure 1, it would be helpful to have error bars included in bar plots.}

  Figure 1 now included error bars (as does the new Figure 6).

\item \emph{There is a typo in Eqn 15.}

  Thank you very much for pointing this out -- this typo is now fixed.

\end{enumerate}

%\bibliographystyle{ims-nourl}
%\bibliography{articles}

\end{document}
