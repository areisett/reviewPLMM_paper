\documentclass{article}
\usepackage[left=0.5in,right=0.5in,bottom=0.75in,top=0.75in]{geometry}
\usepackage{amsmath}
\usepackage{amssymb}
\usepackage{natbib}
\usepackage{color}
\providecommand{\note}[1]{\textcolor{red}{#1}}

\begin{document}

\subsubsection*{Response to comments by Reviewer \#1}

Thank you very much for your detailed reading of the paper and your insightful questions. You raise a number of good points and we have modified the manuscript in several places based upon your feedback.  In our opinion, these revisions have nicely improved upon the original submission.

\begin{enumerate}

\item \emph{This is a nicely conducted review/educational article on an important topic. My main ``complaint'' is that it is not ``sufficiently friendly''.  The article directly jumps into a statistical question (and solutions). There is insufficient discussion/description on the application scenarios. There can/should be clearer descriptions on when the reviewed approaches can/cannot be applied.}

  Response.

\item \emph{In the discussion of methods, there are similar concerns. For example, for some or most statisticians, Lasso is not a new concept. But I can bet that not all practitioners know what Lasso does. I acknowledge that this article cannot and should not review Lasso and other penalizations. But still, it is worthwhile briefly discussing what the key methodological components can do and providing references.}

  Response

\item \emph{Simulation is conducted and is informative. However, for practitioners, it is of interest to see what kind of practical impact the approaches may have. Data examples are strongly recommended.}

  Response

\item \emph{Related to the above points: how can someone not savvy in statistics/penalization etc. use these approaches? Any simple software? I can see some scattered information for example in Discussion. One possibility is to have a separate subsection or table for software. Application examples too.}

  Response

\item \emph{I understand that the authors’ focus is on linear model. But, many if not most genetic studies do not have a continuous outcome and linear model. Can the models/approaches be extended to other outcomes/models with reasonable effort? Either way, it is worth knowing.}

  Response

\item \emph{This may seem contradicting the above points: for a few statisticians, it may be of interest to know more about the statistical properties, especially under high-dimensional settings. Brief discussions (maybe a paragraph? Or a subsection?) on the statistical properties may be informative.}

  Response.

\end{enumerate}

\newpage

\subsubsection*{Response to comments by Reviewer \#2}

Thank you very much for your detailed reading of the paper and your insightful questions. You raise a number of good points and we have modified the manuscript in several places based upon your feedback.  In our opinion, these revisions have nicely improved upon the original submission.

\begin{enumerate}

\item \emph{This is a very insightful paper as to the performance of lasso LMM methods in GWAS analyses and provides nice intuition using theoretical derivations on the interplay between  population structure and environmental heterogeneity. The paper could highly benefit from using more realistic simulation scenarios reflecting standard GWAS (increase the number of SNPs in model which is more in the 100,000s in current GWASes) as well as provide some assessment of how robust the results would be to different lasso penalty parameters (as in practice we would not have a priori an optimal value).}

  Response

\item \emph{In the captions of Figs. 1 and Table 1, it is not clear how many replicates are used to estimate these quantities (e.g. all p=1000 SNPs used in simulations or only the 50 SNPs selected by lasso?). For Figure 1, it would be helpful to have error bars included in bar plots. Also, there is a typo in Eqn 15.}

  Response

\end{enumerate}

%\bibliographystyle{ims-nourl}
%\bibliography{articles}

\end{document}
